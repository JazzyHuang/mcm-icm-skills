\documentclass{mcmthesis}
\mcmsetup{CTeX = false,   % 使用 CTeX 套装时,设置为 true
        tcn = 123456, problem = F,
        sheet = true, titleinsheet = true, keywordsinsheet = true,
        titlepage = false, abstract = true}
% \usepackage{newtxtext,newtxmath}
\usepackage{palatino}
\usepackage{lipsum}
\usepackage{algorithm}
\usepackage{algpseudocode}
\usepackage{subfig}
\usepackage{mdframed}
% \usepackage{amsmath}
% \usepackage{hyperref}
\usepackage{indentfirst}
\usepackage{enumitem}
\usepackage{booktabs}
\usepackage{float}
\usepackage{multirow} % 用于合并单元格
\usepackage{makecell} % 用于换行单元格
\usepackage{amsmath}  % 用于数学符号
\usepackage{array}    % 用于自定义列格式
\usepackage{ragged2e} % 用于文本对齐
\usepackage{geometry} % 调整页面边距以适配内容
\geometry{a4paper, margin=1in}

\setlength{\parindent}{2em}
\makeatletter
\renewcommand\@cite[1]{\textsuperscript{[#1]}}
\makeatother

\title{Changing rules on the virtual battlefield}
% 这一段是备忘录部分,如果题目没有让写备忘录 或者书信 可以不要
% \author{\small \href{http://www.latexstudio.net/}
%   {\includegraphics[width=7cm]{mcmthesis-logo}}}
% \date{\today}

%  \memoto{\LaTeX{}studio}
% \memofrom{Liam Huang}
% \memosubject{Happy \TeX{}ing!}
% \memodate{\today}
% %\memologo{\LARGE I'm pretending to be a LOGO!}

\begin{document}
\begin{abstract}
\setlength{\parskip}{2pt}

As the digitalization process accelerates, cybercrime has become a transnational
threat. Although countries have successively introduced cybersecurity policies, due to
differences in implementation time, resource allocation and national conditions, there
is an urgent need to quantify policy effectiveness and identify best practices through
data-driven methods. This problem aims to reveal the global distribution of cyber-crime,
evaluate the heterogeneous effects of policies in various countries, and analyze
the association mechanism between demographic factors and crime risks, so as to pro-vide
policymakers with a data-driven optimization path.

For Task 1, in order to explore the distribution of cybercrime, first collect data and
clean the data, and then observe the distribution of cybercrime through visualization.
It is found that high-incidence areas of cybercrime activities often roughly overlap with
areas with low GCI scores. Then, a VBGMM model is established to cluster countries
with the same attributes and finally divide them into victim countries and attacking
countries. Finally, a multi-level analysis model based on unordered multi-classification
Logistics regression is constructed. It is found that the success of cybercrime often oc-curs
in areas with complex hacker activities, insufficient technical support and low in-ternational
cooperation, while the thwarting of crimes is mainly seen in environments
with close cooperation and advanced technology. Asset-rich regions are more inclined
to report attacks, while countries with sound legal systems provide effective support,
which enhances the prosecution effect of cybercrime.

For Task 2, in order to evaluate the impact of policies on cybercrime, a multi-period
double difference model was established, and parallelism and placebo tests were con-ducted
to analyze the heterogeneous impact of the five dimensions of policies on the
crime rates of the attacking and victim countries, and finally the policy effects of legal
improvement and technology investment were significant. In order to analyze the time
of implementation of cybersecurity policies in various countries and the temporal dy-namics
of their impact on cybercrime, a survival analysis model was established, and it
was found that for victim countries, policy implementation and duration significantly
reduced the risk of cybercrime.

For Task 3, in order to analyze the association mechanism between demographic
data of different countries and cybercrime, a structural equation model was established
and a fitness test was conducted. According to the model analysis, countries with
high Internet penetration rates and high education levels among victim countries are
more likely to become targets of cybercrime, while cybercrime in attacking countries
is mainly driven by technical capabilities and organizational levels, and the impact of
demographic factors is not significant. 

\begin{keywords}
VBGMM; DID Model; Cox TVPH; SEM
\end{keywords}
\end{abstract}
\maketitle
\vspace{-1.2em}
%% Generate the Table of Contents, if it's needed.
 \tableofcontents
 \newpage
%%
%% Generate the Memorandum, if it's needed.


%%\section为一级标题,\subsection为二级标题 \subsubsection为三级标题

\section{Introduction}
% \begin{itemize}……\end{itemize}为列表文本
\lipsum[2]
\begin{itemize}
% \item为要点强调,表现为黑色圆点
\item minimizes the discomfort to the hands, or 
\item maximizes the outgoing velocity of the ball.
\end{itemize}
We focus exclusively on the second definition.
\begin{itemize}
\item the initial velocity and rotation of the ball,
\item the initial velocity and rotation of the bat,
\item the relative position and orientation of the bat and ball, and
\item the force over time that the hitter hands applies on the handle.
\end{itemize}
\par
\lipsum[3]
\begin{itemize}
\item the angular velocity of the bat,
\item the velocity of the ball, and
\item the position of impact along the bat.
\end{itemize}
\par
\lipsum[4]
% 下述几种格式的效果可以去PDF里查看应用
\emph{center of percussion} [Brody 1986], \lipsum[5]
\begin{Theorem} \label{thm:latex}
\LaTeX
\end{Theorem}
\begin{Lemma} \label{thm:tex}
\TeX .
\end{Lemma}
\begin{proof}
The proof of theorem.
\end{proof}
\begin{figure*}[ht]
  \centering
    \subfloat[subcaption1]{\includegraphics[width = 0.3\textwidth]{mcmthesis-logo.pdf}}
    \hfill
    \subfloat[subcaption2]{\includegraphics[width = 0.3\textwidth]{mcmthesis-aaa-eps-converted-to.pdf}}
    \hfill
    \subfloat[subcaption3]{\includegraphics[width = 0.3\textwidth]{mcmthesis-aaa.eps}} 
  \caption{caption}
  \label{fig:label}
\end{figure*}
  
\subsection{Problem Background}

With the popularization of Internet technology, global connections are becoming in-creasingly
close, which not only improves communication efficiency, but also provides
opportunities for cybercrime and increases the security threats faced by individuals
and organizations. Many cybersecurity incidents cross national borders, making in-vestigations
more difficult. In order to protect their own image, some organizations
choose to conceal the fact that they have been hacked. Although this can avoid neg-ative
publicity in the short term, it weakens the overall awareness and response ca-pabilities
of society to cybersecurity threats. Therefore, countries need to unite and
formulate comprehensive national cybersecurity policies based on their own charac-teristics
to strengthen the cybersecurity protection system.
\begin{itemize}
\item
\item
\item
\item
\end{itemize}
\par
\lipsum[7]

\begin{algorithm}
  \caption{111}
  \begin{algorithmic}[1]
  \State $\text{Lambda} \gets 0.04$
  \State $\text{dr}_0 \gets 0.2$
  \State $t|\text{NR} \gets 0$
  \State $i \gets 0$
  \While{$i < n$}
      \State $r \gets \Call{Random}$
      \State $p \gets \Call{Exponential}{\text{Lambda}, t|\text{NR}}$
      \If{$r \leq p$}
          \State $\text{day}_\text{n} \gets \Call{RandomDays}$
          \For{$j \gets 1$ \textbf{to} $\text{day}_\text{n}$}
              \State $\text{dr}_j \gets 0.15 \cdot r + 0.6$
          \EndFor
      \Else
          \State $\text{dr} \gets \text{dr}_0$
          \State $t|\text{NR} \gets t|\text{NR} + 1$
      \EndIf
      \State $i \gets i + 1$
  \EndWhile
  \end{algorithmic}
  \end{algorithm}

\subsection{Restate of the Problem}

Considering the background information and constraints, developing a National
Security Doctrine to solve the following problems:

1. Analyze the global distribution of cybercrime, identify high-risk countries, and de-termine
where cybercrime is successful, disrupted, reported, and prosecuted.

2. Modeling over time to determine which parts of a policy or law are effective or
ineffective in addressing cybercrime.

3. Identify countries where demographic data is significantly correlated with the dis-tribution
of cybercrime and explain how this supports or challenges established the-ories
\subsection{Our Work}
To solve the problem, we built and optimized the entire model:

For Task 1, we collected and processed data, observed the distribution of cyber-crime
through visualization, and analyzed the characteristics of countries by cluster-ing
them into attacking countries and victim countries in order to find high targets
of cybercrime. We also built a multi-level analysis model based on unordered multi-classification
logistic regression to analyze the law of cybercrime.

For Task 2, we first considered the effectiveness of the policy, and discussed the
effectiveness of the five aspects of the policy on the cybercrime rate of the attacking
country and the victim country by building a multi-period double difference model.
Then, in order to explore the impact of the time of policy adoption on cybercrime, we
established Cox’s time varying proportional hazard model.

For Task 3, we introduced the indicator of demographic data and built a structural
equation model to explore the correlation between cybercrime data and demographic
data in different countries.

Finally, we conducted a sensitivity analysis on the model
\begin{figure}[h]
\small
\centering
\includegraphics[width=12cm]{mindmap.png}
\caption{Mind Map} \label{fig:aa}
\end{figure}

\section{Assumptions and Justifications}
\begin{itemize}
\item Demographic characteristics (such as education index, GDP, human develop-ment
index, etc.) have a significant nonlinear impact on the occurrence of cy-bercrime.

Reason:Education level, economic investment, comprehensive development,
etc. are related to cybersecurity awareness and capabilities, resource allocation,
etc., and the relationship is complex.
\item The impact of each dimension of the GCI index on the occurrence of cybercrime
has different weights.

Reason: The impact of policies and laws, technical facilities, education and train-ing,
international cooperation, etc. on cybercrime in different regions and situa-tions
varies in terms of intensity and mode.
\item There is a threshold effect on the impact of Internet penetration and mobile device
usage on cybercrime.

Reason: The Internet and mobile devices are the main platforms for cybercrime,
but their impact may not be linear. In the early stage, with the popularization
of the Internet and mobile devices, cybercrime may increase due to the increase
in users, but when it reaches a certain saturation, the growth rate of cybercrime
may slow down or decline due to factors such as increased security awareness
and strengthened protective measures.
\item The impact of economic factors (such as GDP) on the occurrence of cybercrime is
interactive.

Reason: GDP not only affects the investment in cybersecurity resources and the
level of technological development, but also interacts with other economic-related
factors such as education level and employment status to jointly influence the
motivation, opportunities and capabilities of cybercrime.
\end{itemize}
% 
\begin{figure}[h]
\small
\centering
\includegraphics[width=12cm]{mcmthesis-aaa.eps}
\caption{aa} \label{fig:aa}
\end{figure}

\lipsum[8] \eqref{aa}
\begin{equation}
  \left[\begin{array}{c}
    \dot{X} \\
    \dot{Y} \\
    \dot{Z}
    \end{array}\right]=\left[\begin{array}{ccc}
    \cos \phi & -\sin \phi & 0 \\
    \sin \phi & \cos \phi & 0 \\
    0 & 0 & 1
    \end{array}\right]\left[\begin{array}{ccc}
    \cos \theta & 0 & \sin \theta \\
    0 & 1 & 0 \\
    -\sin \theta & 0 & \cos \theta
    \end{array}\right]\left[\begin{array}{ccc}
    1 & 0 & 0 \\
    0 & \cos \psi & -\sin \psi \\
    0 & \sin \psi & \cos \psi
    \end{array}\right]\left[\begin{array}{c}
    \dot{x} \\
    \dot{y} \\
    \dot{z}
    \end{array}\right]
\end{equation}  
% \begin{equation}……\end{equation} 这种形式可以实现右编号
\begin{equation}
a^2 \label{aa}
\end{equation}
% \[   \]这种形式无编号
\[
  \begin{pmatrix}{*{20}c}
  {a_{11} } & {a_{12} } & {a_{13} }  \\
  {a_{21} } & {a_{22} } & {a_{23} }  \\
  {a_{31} } & {a_{32} } & {a_{33} }  \\
  \end{pmatrix}
  = \frac{{Opposite}}{{Hypotenuse}}\cos ^{ - 1} \theta \arcsin \theta
\]
\lipsum[9]

\[
  p_{j}=\begin{cases} 0,&\text{if $j$ is odd}\\
  r!\,(-1)^{j/2},&\text{if $j$ is even}
  \end{cases}
\]

\lipsum[10]

\[
  \arcsin \theta  =
  \mathop{{\int\!\!\!\!\!\int\!\!\!\!\!\int}\mkern-31.2mu
  \bigodot}\limits_\varphi
  {\mathop {\lim }\limits_{x \to \infty } \frac{{n!}}{{r!\left( {n - r}
  \right)!}}} \eqno (1)
\]

\section{Notations}
\begin{table}[htbp]
\centering
\caption{Notations and Definitions}
\label{tab:notation}
\begin{tabular}{ll}
\toprule
\textbf{Notations} & \textbf{Definition} \\
\midrule
$h(t \mid x)$ & The instantaneous risk of the covariate $x$ \\
$b_0(t)$ & Baseline hazard rate \\
$Z_i \ (i = 1, 2, 3, 4, 5)$ & Five aspects of policy \\
$X_i \ (i = 1, 2, 3, 4)$ & Factors that measure cybercrime \\
\bottomrule
\end{tabular}
\end{table}


\section{Data Pre-processing}
\subsection{Data Collection}

\begin{table}[H]
\centering
\caption{Data Sources}
\label{tab:notation}
\begin{tabular}{ll}
\toprule
\textbf{Database Name} & \textbf{Database Website} \\
\midrule
Cybercrime Data & https://verisframework.org/index.html \\
National Policies Data & https://www.itu.int/en/ITU-D/Cybersecurity/Pages/CIRTs/ \\
 & List-of-National-CIRTs.aspx\\
Demographic Statistics & https://databank.worldbank.org/home \\

\bottomrule
\end{tabular}
\end{table}

\subsection{Data Cleaning}
For any feature, if the proportion of NA or "UNKNOWN" in the feature exceeds
10%, then remove the feature.

For any Boolean type feature, if the proportion of False values âA˘ Nâ´ A˘ Nexceeds ´
90%, consider whether to remove the feature. This step needs to be decided based on
the specific analysis purpose, because sometimes even a large proportion of False may
contain important information.

For the remaining features with missing values, use linear interpolation to fill them.

For categorical variables (non-numeric variables), use one-hot encoding to convert
them to numerical variables. One-hot encoding converts each categorical value into a
new binary column (0 or 1), where each column represents a possible category. This
encoding method helps machine learning algorithms better process and understand
categorical data.

\section{Task1: Multi-level cybercrime risk assessment and response model}
\subsection{Data Visualization}
\begin{figure}[H]
\small
\centering
\includegraphics[width=12cm]{map1.png}
\caption{Number of crimes reported worldwide from 2005 to 2024} \label{fig:aa}
\end{figure}

\begin{figure}[H]
\small
\centering
\includegraphics[width=12cm]{map2.png}
\caption{2024 GCI Chart} \label{fig:aa}
\end{figure}

Figure 2 shows the high incidence of cybercrime around the world, with particularly
significant clusters in North America and Western Europe. The extreme values
appear in the United States, and the changes in color and height reflect the frequency
and success rate of cybercrime. Some Southeast Asian countries also show high
criminal activity. Figure 3 reflects the global cybersecurity score. North America, Europe
and some East Asian countries have high scores, indicating their investment and
policy support in cybersecurity. High incidence areas of cybercrime activities often
overlap with areas with low GCI scores. For example, Africa and some South Asian
countries not only face low cybersecurity scores, but also report high cybercrime rates.
This shows that there is a direct proportional relationship between cybercrime and inadequate
cybersecurity strategies.

\subsection{Variational Bayesian Gaussian Mixture Clustering}
\subsubsection{Principal component analysis dimensionality reduction}

By searching the VERIS community database, we found that there are 136 indicators
related to cybercrime attacking countries and 825 indicators related to cybercrime
victim countries. Considering that when the number of features is very large, model
training may become complicated and the computational cost is very high, and in
order to remove redundant information, principal component analysis is performed
on the indicator data to reduce the dimensionality, and it is ensured that the selected
principal components can explain at least 90\% of the variability in the original data,
thereby reducing the attacking country indicator to 8 dimensions and the victim country
indicator to 12 dimensions.

\subsubsection{Clustering process and results}

The reason for choosing the variational Bayesian Gaussian mixture model (VBGMM)
for clustering is mainly based on the high complexity and dimensionality of the
data. First, whether it is 8 dimensions or 12 dimensions, it still seems high for the processed
data set, and VBGMM is good at processing high-dimensional data and effectively
avoids overfitting problems through its internal mechanism, which is especially
important for data after dimensionality reduction, ensuring that the effectiveness and
accuracy of clustering results are maintained while simplifying the data structure. Secondly,
due to the lack of prior knowledge about the optimal number of clusters, VBGMM
can adaptively determine the most appropriate number of clusters based on the
intrinsic distribution characteristics of the data, without pre-setting a fixed number of
clusters, which is particularly critical for exploring data sets with unknown structures.
In addition, considering that samples may be distributed on the boundaries between
multiple clusters, VBGMM provides a probability-based method to assign samples to
each cluster. This soft assignment method not only better reflects the true distribution
of the data, but also increases the understanding of data uncertainty. Finally, the regularization
effect introduced by VBGMM helps to improve the generalization ability of
the model, which is particularly suitable for data sets with relatively small sample sizes
but rich features. In summary, given the high complexity and high dimensionality of
the data and the uncertainty requirements for the number of clusters, it is reasonable
to choose VBGMM for clustering.

Since VBGMM assumes that the data is drawn from a multivariate normal distribution,
it is helpful to standardize the data to zero mean and unit variance. This ensures
that all features are on the same scale and avoids some features having a disproportionate
impact on the clustering results due to their large dimensions.

Since the clustering assumption of VBGMM is that the data satisfies a multivariate
normal distribution, it is necessary to consider applying a transformation (such as the
Yeo-Johnson transformation) to make it closer to a normal distribution.

The clustering results obtained using the algorithm in Table 3 are visualized as
shown in Figure 4.

For each subcategory, find the 5 most important features as the representative features
of the category and name them as shown in Figure 5

\begin{table}[H]
\centering
\caption{Survival EM Algorithm}
\label{tab:survival_em}
\begin{tabular}{|p{0.9\textwidth}|} % 单列宽度适配文本
\hline
\textbf{Initialize:} \\
$t \leftarrow 0$ \\
$\boldsymbol{\theta}^{(t)} \leftarrow \boldsymbol{\theta}^0$ \\
$\text{converged} \leftarrow \text{false}$ \\
\hline
\textbf{While} not converged and $t < \text{maxIter}$ \textbf{do:} \\
\quad \textit{E-Step:} Compute Expectation of the complete-data log-likelihood \\
\quad $Q(\boldsymbol{\theta}|\boldsymbol{\theta}^{(t)}) \leftarrow 0$ \\
\quad For $i = 1$ to $n$ do: \\
\quad \quad If $\delta_i = 1$ then: // Complete data \\
\quad \quad \quad $Q(\boldsymbol{\theta}|\boldsymbol{\theta}^{(t)}) \leftarrow Q(\boldsymbol{\theta}|\boldsymbol{\theta}^{(t)}) + \log f(x_i|\boldsymbol{\theta})$ \\
\quad \quad Else: // Censored data \\
\quad \quad \quad $Q(\boldsymbol{\theta}|\boldsymbol{\theta}^{(t)}) \leftarrow Q(\boldsymbol{\theta}|\boldsymbol{\theta}^{(t)}) + \log S(x_i|\boldsymbol{\theta})$ // Contribution from censored observation \\
\quad \quad End If \\
\quad End For \\
\quad \textit{M-Step:} Maximize $Q(\boldsymbol{\theta}|\boldsymbol{\theta}^{(t)})$ with respect to $\boldsymbol{\theta}$ \\
\quad Solve $\frac{\partial Q(\boldsymbol{\theta}|\boldsymbol{\theta}^{(t)})}{\partial \boldsymbol{\theta}} = 0$ for $\boldsymbol{\theta}$ to get $\boldsymbol{\theta}^{(t+1)}$ \\
\quad // Check Convergence \\
\quad If $||\boldsymbol{\theta}^{(t+1)} - \boldsymbol{\theta}^{(t)}|| < \epsilon$ or $|L(\boldsymbol{\theta}^{(t+1)}) - L(\boldsymbol{\theta}^{(t)})| < \epsilon$ then: \\
\quad \quad $\text{converged} \leftarrow \text{true}$ \\
\quad End If \\
\quad $t \leftarrow t + 1$ \\
\quad $\boldsymbol{\theta}^{(t)} \leftarrow \boldsymbol{\theta}^{(t+1)}$ \\
End While \\
Return $\boldsymbol{\theta}^{(t)}$ \\
\hline
\end{tabular}
\end{table}

\begin{figure}[H]
\small
\centering
\includegraphics[width=12cm]{clustering.png}
\caption{Clustering of attacking and victim countries} \label{fig:aa}
\end{figure}

\begin{figure}[H]
\small
\centering
\includegraphics[width=12cm]{clustering_type.png}
\caption{Clustering Type} \label{fig:aa}
\end{figure}

\subsection{Multi-level analysis model based on unordered multi-classification
logistics regression}
When studying the success and failure, reporting and prosecution of cybercrime,
we tend to consider the characteristics of global interconnection and focus on how to
promote international cooperation rather than just focusing on the opposing classifications
between countries. This can better understand the dynamic characteristics of
cybercrime, so we will not discuss them separately by country type here, but take a
unified study to summarize the rules.

In order to comprehensively consider and link factors at different levels, by integrating
specific case analysis at the micro level with factors such as policy environment
and international cooperation at the macro level, we can better understand the
dynamic changes and rules of cybercrime. A multi-level analysis model is established
here.

The first-level model focuses on the results of specific cybercrime events and tries
to find out which factors directly affect these results.

Since there are four categories of dependent variables and there is no natural order
between these states, an unordered multi-classification logistic regression model is
constructed at the first level. Here, the types of physical attacks, the security of different
types of assets, the carriers or ways of risk occurrence, and hacker attack methods
are selected as influencing factors.

The second-level model focuses on higher-level factors, aiming to explore why different
countries or regions have different results in dealing with cybercrime. Referring
to the evaluation criteria of the GCI index, the five major aspects of law, technology,
organization, capacity building and cooperation are used to measure the policies of
each country, and according to the relevant indicators of each aspect listed by the International
Telecommunication Union, data is collected and the scores of each aspect
are obtained by quantitative synthesis.

In order to ensure the independence of the independent variables in the regression
model, the coefficient estimation instability, standard error increase and significance
test result distortion caused by the high correlation between the independent variables
are avoided, so as to ensure the accuracy and explanatory power of the model. Before constructing the model, it is necessary to conduct a collinearity test on each layer of
variables. The variance inflation factor is less than 10, which is within a reasonable
range.

Applying the above theory, the model equation is constructed as follows:

First layer:
\setcounter{equation}{0} %
\begin{align}
\log\left(\frac{P(Y=1)}{P(Y=0)}\right) &= \beta_1 - 6.684 \times X_1 + 2.593 \times X_2 - 11.348 \times X_3 - 6.405 \times X_4 \\
\log\left(\frac{P(Y=2)}{P(Y=0)}\right) &= \beta_2 - 0.698 \times X_1 + 11.218 \times X_2 - 16.351 \times X_3 - 11.562 \times X_4 \\
\log\left(\frac{P(Y=3)}{P(Y=0)}\right) &= \beta_3 - 0.473 \times X_1 - 0.348 \times X_2 + 0.828 \times X_3 - 0.131 \times X_4
\end{align}

Among them, $Y=0,1,2,3$ represents the four results of cybercrime success, thwarted, reported and prosecuted, $X_i(i=1,2,3,4)$ represents the four influencing factors of physical attack type, security of different types of assets, carrier or path of risk occurrence, and hacker attack method, $\beta_i(i=1,2,3)$ corresponds to the three results of cybercrime thwarted, reported, and prosecuted.

Second layer:

\begin{align}
\beta_1 &= 0.337 + 0.324 \times Z_1 + 0.087 \times Z_2 - 0.288 \times Z_3 + 0.035 \times Z_4 - 0.147 \times Z_5 \\
\beta_2 &= -0.062 + 0.245 \times Z_1 + 0.051 \times Z_2 - 0.147 \times Z_3 + 0.044 \times Z_4 - 0.121 \times Z_5 \\
\beta_3 &= 0.429 + 0.141 \times Z_1 + 0.048 \times Z_2 - 0.182 \times Z_3 + 0.719 \times Z_4 - 0.259 \times Z_5
\end{align}

Among them, $Z_i(i=1,2,3,4,5)$ represents the five factors of law, technology, organization, capacity building and cooperation.

Analyzing the results of the above model, we can summarize the laws of cybercrime
as follows:
\begin{itemize}
\item Where cybercrime is successful: characterized by highly sophisticated hacker activity,
fewer assets exposed to potential attackers, low levels of international cooperation,
and insufficient technical support. These characteristics provide cybercriminals
with a relatively loose operating space.
\item Where cybercrime is thwarted: characterized by high levels of cooperation (international
or regional), advanced technological development, and a strong focus
on cybersecurity, while avoiding over-reliance on legal means. This means taking
more direct and technical defensive measures to prevent and respond to cyberattacks.
\item Where cybercrime is reported: characterized by having more assets, even in the
face of strong hacker attacks, they may choose to report rather than conceal because
they have better error management mechanisms. This suggests that in
asset-rich environments, organizations and individuals may be more willing to
publicly acknowledge attacks and seek external assistance to solve problems.
\item  Where cybercrime is prosecuted: characterized by the degree of sophistication of
technical infrastructure and the effectiveness of the legal system play a decisive
role, especially when a country or region’s legal system can effectively support
the investigation and trial process of cybercrime. This can greatly increase the
possibility of bringing criminals to justice.
\end{itemize}

\section{Task2:Cybersecurity policy effectiveness evaluation model}
\subsection{Multi-period double difference model}
\subsubsection{Composite Sampling}
Since countries can be divided into countries that attack cybercrime and countries
that are victims, and these two categories of countries face different challenges and
needs in cybercrime, they should be discussed separately so as to adopt targeted strategies.
Considering that victim countries and attacking countries are divided into
three categories according to various characteristics, and the number of countries in
each category is different, a mixed sampling method is used to select representative
countries from each category for research (the formula is as follows), ensuring that
representative samples are drawn from each category in proportion and a fixed number
of samples are added to each category. This method not only ensures that the large
categories are sufficiently representative, but also allows small categories to receive
full attention due to their uniqueness, thereby improving the overall diversity of the
sample and the accuracy of the research results.

\begin{equation}
n_i = \left\lfloor \frac{C_i}{\sum_{j=1}^m C_j} \times (N - n \times k) \right\rfloor + k
\end{equation}

Where $N$ is the total number of samples drawn, where $N=30$, $n_i$ represents the total number of samples selected from the $i$th cluster, $k$ represents the number of additional fixed samples selected from each cluster, where $k=5$, $C_i$ represents the total number of countries in the $i$th cluster, $m$ represents the number of clusters, where $m=3$.

The design takes 30 victim countries and attacking countries for research respectively,
and uses the above mixed sampling method to obtain 18, 6, and 6 representative
countries for the three subcategories of victim countries, and 7, 17, and 6 representative
countries for the three subcategories of attacking countries for targeted and specific research.
\subsubsection{Model construction}
Since the global cybercrime governance policy is implemented in stages and there
are differences in the implementation time of the policy, a multi-period double difference
model is constructed based on econometric theory and research methods to
empirically analyze the inhibitory effect of cybercrime governance policies on the situation
of cybercrime around the world, and further explore the relationship between
various policy indicators and cybercrime. The panel data selected data from 2005 to
2022 for 18 years, in which the experimental group and the control group respectively
covered different countries and regions of cybercrime attacking countries and victim
countries. In order to improve and ensure the accuracy of the research and the accuracy
of the data, the sample data of individual countries with serious data missing were



\begin{table}[H]
\centering
\caption{Selection of Representative Countries}
\label{tab:representative_countries}
\small % 调整字体大小以适配表格宽度
\begin{tabular}{>{\raggedright\arraybackslash}p{0.3\textwidth} >{\raggedright\arraybackslash}p{0.6\textwidth}}
\toprule[1pt]
\textbf{Country Type} & \textbf{Representative Countries} \\
\midrule
Security & UZ, MA, RS, BW, GH, GR, TN, TL, KH, GT, MZ, ME, IQ \\
Vulnerabilities and Equipment Issues & SI, MR, DZ, KW, TZ \\
\midrule
Software Issues and Service Disruptions & SE, HK, SG, NL, ID, VE \\
\midrule
Financial Losses and Business Impact & FR, BE, AE, IE, ZA, TW \\
\midrule
Motivation-driven Attacks & IN, ID, VN, NZ, KR, CA, UA \\
\midrule
Physical Attacks and Control Breaches & VE, JO, AZ, MY, CL, EG, TN, MR, PH, SG, AF, LY, JP, IT, IS, AL, BZ \\
\midrule
Hacker Techniques and Advanced Attacks & IR, PK, TR, SY, AR, CN \\
\bottomrule[1pt]
\end{tabular}
\end{table}
\noindent eliminated. Finally, 60 countries (30 attacking countries and 30 victim countries) were
used as research samples to explore the impact of the implementation of cybercrime
governance policies on the situation of cybercrime and the driving effect of different
policies on the cybercrime prevention and control capabilities of surrounding countries.
The multi-period DID model constructed in this paper is as follows:

\begin{equation}
crime_{it} = \alpha - \beta_1 \times did_{it} + \beta_2 \times post_{it} + u_i + \lambda_t + \varepsilon_{it}
\end{equation}

In the above formula, $i$ represents the country, $t$ represents the year; $crime_{it}$ is the explained variable, i.e., the cybercrime rate of country $i$ in year $t$; $\alpha$ is the constant term; $\beta_1$ is the policy effect, $did_{it}$ is the core explanatory variable, which is the interaction term of the country dummy variable and the time dummy variable, and is used to represent the dummy variables of different GCI indicators; $\beta_2$ is the coefficient, $post_{it}$ is the control variable at the cybercrime level; $u_i$ represents the fixed effect of country $i$, $\lambda_t$ represents the time fixed effect of year $t$; $\varepsilon_{it}$ is the random disturbance term of the multi-period DID regression model.

\subsubsection{Parallel trend test}
Before using the multi-period DID method for analysis and testing, ensure that the
time trends of the cybercrime rates of the treatment group and the control group are
basically parallel, and there is no significant difference. Based on the above work,
we first conduct a parallel test, taking the legal aspect of the attacking country as an
example, the results are as follows:

\begin{figure}[H]
\small
\centering
\includegraphics[width=12cm]{parallel_trend_test.png}
\caption{Parallel Trend Test} \label{fig:aa}
\end{figure}

\subsubsection{Model results}
The model results obtained by solving are as follows:

For the attacking country:
\begin{align}
\text{Law: } \quad crime_{it} &= 0.1927 - 0.5413 \times did_{it} + 0.4440 \times post_{it} + u_i + \lambda_t + \varepsilon_{it} \\
\text{Technology: } \quad crime_{it} &= 0.3593 + 0.6381 \times did_{it} + 0.3069 \times post_{it} + u_i + \lambda_t + \varepsilon_{it} \\
\text{Organization: } \quad crime_{it} &= 0.3152 + 0.327 \times did_{it} + 0.3182 \times post_{it} + u_i + \lambda_t + \varepsilon_{it} \\
\text{Capacity Building: } \quad crime_{it} &= 0.2833 - 0.3080 \times did_{it} + 0.4152 \times post_{it} + u_i + \lambda_t + \varepsilon_{it} \\
\text{Cooperation: } \quad crime_{it} &= 0.3267 - 0.4958 \times did_{it} + 0.2256 \times post_{it} + u_i + \lambda_t + \varepsilon_{it}
\end{align}

For the victim country:
\begin{align}
\text{Law: } \quad crime_{it} &= 0.3528 - 0.4303 \times did_{it} + 0.5372 \times post_{it} + u_i + \lambda_t + \varepsilon_{it} \\
\text{Technology: } \quad crime_{it} &= 0.2493 - 0.3752 \times did_{it} + 0.2638 \times post_{it} + u_i + \lambda_t + \varepsilon_{it} \\
\text{Organization: } \quad crime_{it} &= 0.2904 + 0.3995 \times did_{it} + 0.3672 \times post_{it} + u_i + \lambda_t + \varepsilon_{it} \\
\text{Capacity Building: } \quad crime_{it} &= 0.3632 - 0.4165 \times did_{it} + 0.6330 \times post_{it} + u_i + \lambda_t + \varepsilon_{it} \\
\text{Cooperation: } \quad crime_{it} &= 0.2691 - 0.5297 \times did_{it} + 0.5285 \times post_{it} + u_i + \lambda_t + \varepsilon_{it}
\end{align}
\subsubsection{Placebo test}
After solving the model and obtaining preliminary results, in order to exclude the
impact of non-policy factors on cybercrime and avoid subjective changes in the research
subjects due to advance knowledge of the policy implementation signal, which
may lead to errors in the "policy effect", the following placebo test is conducted:

\begin{figure}[H]
\small
\centering
\includegraphics[width=12cm]{placebo_test.png}
\caption{Placebo test} \label{fig:aa}
\end{figure}
The kernel density distribution of the placebo test results is shown in the figure. It
can be seen that the p value distribution is close to the normal distribution, and the
regression result is not significant, indicating that the impact of the implementation
of legal policies on cybercrime governance on the attacking country is not accidental,
which further verifies the robustness of the model.

\subsubsection{Result Analysis}
After solving the multi-period DID model of the degree of influence of policies on
cybercrime in the attacking country and the victim country, the following radar chart
about $\beta_1$ (policy effect) is drawn:

\begin{figure}[H]
\centering
\includegraphics[width=12cm]{radar.png}
\caption{$\beta_1$ Policy Effect Radar Chart} \label{fig:aa}
\end{figure}

By evaluating the impact of different countries’ policies on cybercrime in five dimensions,
the effectiveness of national policies in cybersecurity is intuitively demonstrated.
In terms of national strategic priorities, attacking countries may prioritize
the development of technical defense and legal deterrence to consolidate their dominance
in cyberspace, while victim countries are forced to selectively ignore certain
dimensions (such as international cooperation) due to limited resources; in terms of
data and execution capabilities, high 1 countries usually have more complete cybercrime
databases and efficient execution agencies, while developing countries with low
1 countries have data gaps or decentralized agencies, and policy effects are difficult to
show.

The legal and technical policies of the attacking country can effectively curb the
occurrence of cybercrime, indicating that it has a complete cybersecurity legal system
(such as the "Cybersecurity Law" and the "Data Protection Law"), and clarifies crossborder
jurisdiction and criminal liability. Deploy advanced defense technologies (such
as AI-driven threat detection and quantum encryption), and the protection level of critical
infrastructure is high; but the policy effectiveness at the capability development
level is low, indicating that they have strong dependence on international cooperation
but weak voice, and it is difficult to obtain threat intelligence in a timely manner.
These countries should avoid technological hegemony and open source their defense
technologies or license them at low prices to countries in need (such as providing free
vulnerability scanning tools). The EU can also fund African countries to build security
infrastructure through the "Digital Europe Plan".

The victim countries’ cooperation policies can effectively curb the occurrence of cybercrime,
indicating that they may dominate the international cybersecurity alliance
and share attacker IP and malicious code feature libraries; but the effectiveness of the
technical level policy is not high, which may be due to the reliance on outdated firewalls
and antivirus software and the lack of real-time monitoring capabilities. These
countries can promote the Zero Trust architecture, force key industries to deploy intrusion detection systems (IDS) or adopt "lightweight" solutions, such as AI-based automated
threat detection (low cost, high coverage) and other measures to improve the
effectiveness of policies at the technical level.

Therefore, the attacking country needs to shift from a "technology exporter" to a
"rule co-builder" to avoid the capability gap that exacerbates the imbalance of global
cybercrime, while the victim countries should focus on resources to make up for
shortcomings (such as laws and technology) and obtain external support through international
cooperation. Only through multi-dimensional policy optimization and innovation
of cooperation mechanisms can a more balanced and sustainable global cybersecurity
ecosystem be built.

\subsection{Cox’s time varying proportional hazard model}
\subsection{principle}
Since the topic requires analyzing the implementation time of cybersecurity policies in various countries and the time dynamics of their impact on cybercrime, and Cox’s time varying proportional hazard model is a statistical model widely used in survival analysis, especially for dealing with time-dependent covariates, this topic innovatively uses this model for research.

\begin{equation}
h(t|x) = \underbrace{b_0(t)}_{\text{baseline}} \cdot \underbrace{\exp\left( \sum_{i=1}^{n} \beta_i \left( x_i(t) - \bar{x}_i \right) \right)}_{\substack{\text{log--partial hazard} \\ \text{partial hazard}}}
\end{equation}

where $h(t|x)$ represents the instantaneous risk (or hazard rate) of a given covariate $x$ at time $t$,
\begin{itemize}
    \item $b_0(t)$: baseline hazard rate,
    \item $\beta_i$: The coefficient of the $i$th covariate, represents the hazard rate when all covariates are zero.
    \item $x_i(t)$: The value of the $i$th covariate at time $t$
    \item $\bar{x}_i$: The average value of the $i$th covariate.
\end{itemize}

\subsection{Hypothesis Testing}
The most important assumption in the Cox model is the proportional hazard assumption, that is, the effect of covariates on survival time does not change over time. Graphical tests (observing whether the survival curves are parallel) show that the survival probability of the "Rejection" group is higher than that of the "Adoption" group, with the former having a longer median survival time of about 14 years, while the latter has a median survival time of about 14 years. In addition, the Log-rank test ($p = 0.0013 < 0.05$) shows that there is a significant difference in the survival probability between the two groups.

\[
h(t \mid x) = h_0(t) \exp(-1.33x_1 - 0.61x_2 + 0.16x_3 + 1.01x_4 - 0.06x_5 + 0.06x_6 + 0.54x_7) \tag{20}
\]

\begin{figure}[H]
\centering
\includegraphics[width=12cm]{curve.png}
\caption{Kaplan-Meier survival curve} \label{fig:aa}
\end{figure}
\subsection{Model Building}
The model equation is established as

For the attacking country:
\[
h(t \mid x) = h_0(t) \exp(-1.33x_1 - 0.61x_2 + 0.16x_3 + 1.01x_4 - 0.06x_5 + 0.06x_6 + 0.54x_7) \tag{21}
\]
\begin{figure}[H]
\centering
\includegraphics[width=12cm]{attack.png}
\caption{Kaplan-Meier survival curve} \label{fig:aa}
\end{figure}
For the victim country:
\[
h(t \mid x) = h_0(t) \exp(-0.30x_1 + 0.08x_2 + 0.92x_3 + 0.10x_4 - 0.86x_5 + 0.07x_6 - 0.13x_7) \tag{22}
\]

where \(x_i (i = 1,2,3,4,5,6,7)\) represents the policy regarding capabilities, cooperation, law, organization, whether the policy is implemented, policy duration, and technology
\begin{figure}[H]
\centering
\includegraphics[width=12cm]{victim.png}
\caption{Kaplan-Meier survival curve} \label{fig:aa}
\end{figure}

The Wald test and likelihood ratio test were used to evaluate the significance of
the regression coefficients in the Cox model: the Wald test calculated the z value by
comparing the regression coefficient with its standard error (absolute value >1.96 indicated
significance at the 0.05 level), while the likelihood ratio test determined whether
the parameter made a meaningful contribution to the model by calculating the p value
(<0.05 indicated a significant improvement in model fit).

\subsection{Model Analysis}
For the attacking country, it can be seen from the equation that technology and
organization contribute more to the occurrence of cybercrime, which means that the
stronger the technical strength and organizational structure, the higher the probability
of the attacking country to launch cybercrime. In contrast, capability and cooperation
play a role in reducing cybercrime. And the duration of the policy slightly increases
the risk, which may be because attackers find ways to bypass the policy over time.

For the victim country, law is one of the most influential factors, indicating that a
stronger legal framework may increase the likelihood of reporting or identifying cybercrime
rather than directly reducing crime. Policy implementation has a significant
effect on reducing cybercrime, which may be because effective policies can prevent
or combat cybercrime. And the duration of the policy also helps to further reduce
the risk, indicating that long-term and effective cybersecurity policies can effectively
prevent cybercrime.

\section{Task3:CyberDemographic Resilience Model}
In addition to considering the five aspects of national policies, namely, legal, technical,
organizational, capacity building and cooperation, demographic data are introduced,
namely, GDP per capita, number of secure Internet servers, and higher education
enrollment rate, which shows that cybercrime is jointly affected by multiple
factors, and there may be complex interactions between these factors. Therefore, a
structural equation model (SEM) is established, which considers the direct and indirect
effects between multiple dependent and independent variables at the same time,
\subsection{Principle}
The connection between variables in SEM is represented by structural parameters,
which provide constants for the invariance of causal relationships between variables
and describe the relationship between observed variables, between observed variables
and latent variables, and between latent variables. These variables can be summarized
into two models, namely measurement models and structural models.

1. Measurement Model mainly represents the relationship between observed variables
and latent variables. The measurement model generally consists of two equations,
which respectively define the relationship between the endogenous latent variable
$\eta$ and the endogenous observed variable $y$, and between the exogenous latent
variable $\xi$ and the exogenous observed variable $x$. The model form is:

\begin{align}
x &= \Lambda_x \xi + \delta \tag{23} \\
y &= \Lambda_y \eta + \varepsilon \tag{24}
\end{align}

Among them, $x$ is a vector composed of exogenous observed variables; $y$ is a vector
composed of endogenous observed variables; $\Lambda_x$ represents the relationship
between exogenous observed variables and exogenous latent variables, which is the
factor loading matrix of exogenous observed variables on exogenous latent variables;
$\Lambda_y$ represents the relationship between endogenous observed variables and
endogenous latent variables, which is the factor loading matrix of endogenous observed
variables on endogenous latent variables; $\delta$ is the measurement error of the
exogenous observed variable $x$; $\varepsilon$ is the measurement error of the
endogenous observed variable $y$; $\xi$ and $\eta$ are the latent variables of $x$
and $y$, respectively.

2. Structural Equation Model mainly represents the relationship between latent
variables. It stipulates the causal relationship between the assumed exogenous latent
variables and endogenous latent variables in the system under study, and the model
form is:

\begin{equation}
\eta = \beta \eta + \Gamma \xi + \zeta \tag{25}
\end{equation}

Among them, $\eta$ is the endogenous latent variable; $\xi$ is the exogenous latent
variable; $\beta$ is the coefficient matrix of the endogenous latent variable $\eta$,
which is also the path coefficient matrix between endogenous latent variables;
$\Gamma$ is the coefficient matrix of the exogenous latent variable $\xi$, which is
also the path coefficient matrix from exogenous latent variables to the corresponding
endogenous latent variables; $\zeta$ is the residual term, representing the part that
cannot be explained by the model.

\subsection{Model establishment}

In combination with the research questions, the basic model diagram of each variable
is established, and by consulting the literature, the policy and demographic data,
as well as the research hypothesis relationship of cybercrime are determined.

For the victim country:

H1: Policy has a significant positive relationship with cybercrime

H2: Demographic data has a significant positive relationship with cybercrime

For the attacking country:

H3: Policy has a significant positive relationship with cybercrime

H4: Demographic data has a significant positive relationship with cybercrime

Based on the above theoretical assumptions, the structural equation model diagram
that affects cybercrime is constructed with the help of the software AMOS, as shown
below.

\begin{figure}[H]
\centering
\includegraphics[width=12cm]{structure.png}
\caption{Structural equation model diagram of the impact of cybercrime} \label{fig:aa}
\end{figure}

\subsection{Fitness test}
The established model was tested for fitness through AMOS, and the following
table was obtained (due to space limitations, only the fitness test of the victim country
is listed, but the attacking country also passed the fitness test).

\begin{table}[htbp]
  \centering
  \caption{Victim Country Fit Test}
  \label{tab:victim-fit}
  \begin{tabular}{cccc}
    \toprule[1.2pt]
    \textbf{Fit Test Index} & \textbf{Fit Standard} & \textbf{Model Result} & \textbf{Fit Evaluation} \\
    \midrule[0.8pt]
    CMIN   & 1$\sim$3     & 1.774 & Good \\
    RMSEA  & $<0.08$      & 0.083 & Acceptable \\
    RMR    & $<0.08$      & 0.085 & Acceptable \\
    GFI    & $>0.90$      & 0.927 & Good \\
    CFI    & $>0.90$      & 0.949 & Good \\
    IFI    & $>0.90$      & 0.944 & Good \\
    TLI    & $>0.90$      & 0.985 & Good \\
    \bottomrule[1.2pt]
  \end{tabular}
\end{table}

\subsection{Model analysis}
According to the victim country model, demographics have a significant positive
impact on cybercrime. Therefore, it can be inferred that among victim countries, countries
with high Internet penetration, developed economy, and high education level are
more likely to become targets of cybercrime. For example, developed countries such
as Canada and South Korea may face more cyberattacks due to their well-developed
Internet infrastructure and frequent economic activities. In the attacking country model,
demographics do not have a significant predictive effect on cybercrime. This may
mean that cybercrime in attacking countries is more driven by technical capabilities
and organizational levels rather than traditional demographic factors. Therefore, attacking
countries may be those countries that have high Internet penetration and developed
economy but strong legal and technical defense capabilities, such as China
and Russia, which may become the source of cyberattacks due to the concentration of
technical resources.

According to the victim country model, high Internet penetration and education
level may lead to more cybercrime victims, which supports the theory that "countries
with high digitalization are more likely to become targets." However, this may also
confuse the theory because high education level is usually associated with higher cybersecurity
awareness, but the data show that education level positively affects crime
victimization, which may be because countries with high education levels use more
online services, increasing the risk of exposure.

For the attacking countries, the demographic data is not significant, which may
support the theory that "attack behavior relies more on technical and organizational
factors", but the problems of data quality or indicator selection need to be eliminated.
For example, the demographic indicators (secure Internet servers, higher education
enrollment rate) in the attacking country model may not effectively reflect the actual
influencing factors, and other indicators such as the proportion of technical practitioners
or dark web activity may need to be introduced.

In view of the cybercrime governance needs of different countries, the following
policy recommendations are put forward: For the victim countries, the access rights
to high-value data should be restricted by promoting the Zero Trust architecture to reduce
the risk of exposure. At the same time, compulsory cybersecurity courses should
be added in higher education to balance digital convenience and risk awareness, and
security subsidies should be provided to small and medium-sized enterprises to alleviate
the problem of insufficient resources, so as to systematically improve defense
capabilities. For the attacking countries, it is necessary to strictly control the export of
vulnerability scanning tools and encryption technologies to prevent the abuse of technology,
require transparent attack tracing through international cooperation mechanisms
such as the Budapest Convention to increase the cost of crime, and introduce
alternative indicators such as dark web activity and the number of APT organizations
to more accurately reflect the characteristics of attack behavior. At the international
level, we should promote the establishment of a unified cybercrime data reporting
framework to reduce analytical bias. At the same time, we should establish a global
cybersecurity fund to support technology transfer and capacity building in developing
countries, promote cross-border collaboration and technology sharing, and build a
fairer and more transparent global cyber governance system.
\section{Analysis on Model’s Sensitivity}
We used five factors of national policy and data on cybercrime when constructing
a multi-level analysis model. However, in actual calculations, due to national statistics
or publication errors, the data is often inaccurate and fluctuates, which may affect the
results to a certain extent. Therefore, in order to test whether our model is still stable
when external conditions are disturbed, we use sensitivity analysis to evaluate our
model. In order to simulate data fluctuations of different magnitudes, we added 2\%,
5\% and 10\% disturbances to the data, which were brought into the prediction model
for calculation and compared with the original situation without disturbance. Here,
we take the frequency of physical attacks as an example,

\begin{figure}[H]
\centering
\includegraphics[width=12cm]{sensitivity analysis.png}
\caption{Sensitivity analysis} \label{fig:aa}
\end{figure}
By changing the frequency of physical attacks in the multi-layer analysis model,
we calculate the results of the log-odds ratio, thereby analyzing the sensitivity of the
factor of physical attack frequency. From the figure, we can see that with the increase of
attack frequency, the trend of log-odds ratio changes little, indicating that our equation
is stable.

\section{Strengths and Weaknesses}

\subsection{Strengths}
1. The DID model can effectively control unobserved time-invariant characteristics
and avoid bias introduced by fixed individual effects. And because the model compares
the time changes of the treatment group and the control group, it can reduce the
impact of selection bias and extract causal relationships. It is suitable for various types
of policy evaluation and intervention analysis and can handle different time series
data.

2. SEM can handle multiple dependencies at the same time and is suitable for analyzing
direct and indirect effects between variables. Concepts can be represented by
latent variables (unobserved variables) to improve the explanatory power of the model,
especially in social science research, which is often used to quantify characteristics
that are difficult to measure directly.

\subsection{Weaknesses}
1. Although the DID model can control fixed effects, the model may still be affected
by unobserved time-varying factors, which may affect both the treatment group and
the control group. Multi-period DID requires a large amount of panel data to ensure
the reliability of the results, and data collection and processing may be more complicated.
And if an exogenous shock occurs during the analysis period (such as an economic
crisis, an epidemic, etc.), it may affect the results, making causal relationships more
difficult to identify

2. Structural equation models require a larger sample size to ensure the stability
and reliability of parameter estimates, and the model assumptions are stricter. SEM
assumes a linear relationship between variables, which may not apply to all actual
situations, leading to model bias, and is more sensitive to missing data and outliers,
which may affect the accuracy of the results.

\newpage
\phantomsection
\addcontentsline{toc}{section}{References}
\begin{thebibliography}{99}
\bibitem{1} Padyab M ,Padyab A ,Rostami A , et al.Cybercrime in Nordic countries: a scoping
review on demographic, socioeconomic, and technological determinants[J].SN
Social Sciences,2024,4(11):205-205.
\bibitem{2} Meiling C ,Yaqin S ,Jinping L , et al.DRKPCA-VBGMM: fault monitoring
via dynamically-recursive kernel principal component analysis with
variational Bayesian Gaussian mixture model[J].Journal of Intelligent
Manufacturing,2022,34(6):2625-2653.
\bibitem{3} Wang H ,Ge T ,Wang Q , et al.A Mathematical Prediction Model for Postoperative
Infection Based on Logistic Multiple Regression Analysis in the Assessment
of Surgical Outcome and Prediction of Infection in Elderly Spinal Fractures[J].Alternative
Therapies in Health and Medicine,2024,30(7):179-183.
\bibitem{4} Wang Jin, Li Junyan, Wu Jianlong. The Impact of Big Data Comprehensive Pilot
Zones on Regional Scientific and Technological Innovation CapabilityâA˘TPolicy ˇ
Effect Evaluation Based on Multi-period Difference-in-Differences [J/OL]. Studies
in Science of Science, 1-21 [2025-01-28].
\bibitem{5} Parveen K ,Phuc B Q T ,Alghamdi A A , et al.Author Correction: Unraveling
the dynamics of ChatGPT adoption and utilization through Structural Equation
Modeling.[J].Scientific reports,2024,14(1):28208.
\bibitem{6} \url{http://www.chinatex.org/}
\end{thebibliography}

\newpage
\phantomsection
\addcontentsline{toc}{section}{MEMO}
\noindent \textbf{\Large MEMO}

\vspace{1em}
\noindent
\textbf{To:} ITU Cybersecurity Summit National Leaders \\
\textbf{From:} Team 2504223 \\
\textbf{Date:} January 28, 2025 \\
\textbf{Subject:} National security policy making on cybercrime

\vspace{1em}

\noindent Dear leaders of the country:

We present this memorandum to you with urgency to share our latest research
results on global cybercrime governance and provide data-driven policy
recommendations for the upcoming Cybersecurity Summit. As the digitalization
process accelerates, cybercrime has become a transnational threat. Its
complexity and concealment make cross-border law enforcement and policy
coordination difficult, seriously threatening national security and economic
stability. Although countries have successively introduced cybersecurity
policies, due to differences in implementation time, resource allocation and
national conditions, it is urgent to quantify policy effectiveness and identify
best practices through scientific methods. The policy effect evaluation shows that the improvement of laws and technology
investment significantly reduce the rate of cybercrime, and the effect is
significant within 1--2 years after the implementation of the policy. Therefore,
it is recommended to implement the strategy of ``technical defense first, legal
deterrence follow-up'' in the terms of demographic correlation factors. Countries
with high Internet penetration and high education levels are more likely to
become targets of attacks, while in attacking countries, demographic factors
(such as economy and education) have no significant impact on attack behavior,
and the main driving factors are technical capabilities and organizational
levels. Based on the above findings, the recommendations for victim countries include
promoting zero-trust architecture to reduce exposure risks, limiting access
rights to high-value data, and adding cybersecurity courses in higher education,
while providing security subsidies for small and medium-sized enterprises. For
attacking countries, it is recommended to strictly control the export of
vulnerability scanning tools and encryption technologies to prevent the abuse
of technology, and promote the transparency of attack tracing through
mechanisms such as the Budapest Convention to increase the cost of crime. At the international level, a global cybersecurity framework such as the
expansion of VERIS should be established to unify data standards and reduce
statistical bias, and a capacity building fund should be established under the
leadership of developed countries to support technology transfer and talent
training in developing countries. Cybercrime governance must take account of
differences in national digital risks and shared international responsibilities.
Victim countries should balance digital risks through technological defense
and education investment; attacking countries need to replace technological
hegemony with transparency and cooperation; and the international community
needs to build an inclusive framework to promote resource sharing and rule-making.
This concludes our project. Thank you.

\vspace{2em}

\begin{flushright}
Yours Sincerely,\\
Team \#2504223
\end{flushright}

\begin{appendices}

%  备忘录正文部分
%  \begin{memo}[Memorandum]
% 	\lipsum[1-3]
% \end{memo}

\section{First appendix}

\lipsum[13]

Here are simulation programmes we used in our model as follow.\\

\textbf{\textcolor[rgb]{0.98,0.00,0.00}{Input matlab source:}}
\lstinputlisting[language=Matlab]{./code/mcmthesis-matlab1.m}

\section{Second appendix}

some more text \textcolor[rgb]{0.98,0.00,0.00}{\textbf{Input C++ source:}}
\lstinputlisting[language=C++]{./code/mcmthesis-sudoku.cpp}

\end{appendices}
\newpage

\section*{\centering Report on Use of AI}
\subsection*{OpenAI-GPT5-Thinking}
\textbf{Query1 :} prompt1  \par
\textbf{Output1:} 
\begin{mdframed}[
    linewidth=2pt,        % 边框线粗细
    linecolor=black,        % 边框颜色
    innertopmargin=10pt,    % 框内顶部间距
    innerbottommargin=10pt, % 框内底部间距
    innerleftmargin=10pt,   % 框内左侧间距
    innerrightmargin=10pt,  % 框内右侧间距
    skipabove=5pt,         % 框上方间距
    skipbelow=5pt,          % 框下方间距
    backgroundcolor=gray!20
]
AI Output
\end{mdframed}


\end{document}

%% 
%% This work consists of these files mcmthesis.dtx,
%%                                   figures/ and
%%                                   code/,
%% and the derived files             mcmthesis.cls,
%%                                   mcmthesis-demo.tex,
%%                                   README,
%%                                   LICENSE,
%%                                   mcmthesis.pdf and
%%                                   mcmthesis-demo.pdf.
%%
%% End of file `mcmthesis-demo.tex'.

