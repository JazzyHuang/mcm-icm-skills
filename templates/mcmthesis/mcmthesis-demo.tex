% MCM/ICM LaTeX Template
% Based on mcmthesis class
% Adapted for MCM/ICM Automation System

\documentclass[12pt]{article}

% === 基本设置 ===
\usepackage[utf8]{inputenc}
\usepackage[T1]{fontenc}
\usepackage{amsmath,amssymb,amsfonts}
\usepackage{graphicx}
\usepackage{booktabs}
\usepackage{hyperref}
\usepackage{xcolor}
\usepackage{geometry}
\usepackage{fancyhdr}
\usepackage{lastpage}
\usepackage{float}
\usepackage{caption}
\usepackage{subcaption}
\usepackage{algorithm}
\usepackage{algorithmic}
\usepackage{listings}
\usepackage{natbib}

% === 页面设置 ===
\geometry{
    a4paper,
    left=2.5cm,
    right=2.5cm,
    top=2.5cm,
    bottom=2.5cm
}

% === 页眉页脚 ===
\pagestyle{fancy}
\fancyhf{}
\lhead{Team \#XXXXX}
\rhead{Page \thepage\ of \pageref{LastPage}}
\renewcommand{\headrulewidth}{0.4pt}

% === 控制号设置 ===
\newcommand{\controlnumber}{XXXXX}
\newcommand{\problemtype}{A}

% === 标题页 ===
\title{\Large\textbf{Problem \problemtype: [Your Title Here]}}
\author{Team \#\controlnumber}
\date{February 2026}

% === 摘要环境 ===
\newenvironment{summary}{
    \begin{center}
    \textbf{\large Summary}
    \end{center}
    \begin{quotation}
}{
    \end{quotation}
    \vspace{1em}
}

% === 代码样式 ===
\lstset{
    basicstyle=\ttfamily\small,
    breaklines=true,
    frame=single,
    language=Python,
    numbers=left,
    numberstyle=\tiny
}

\begin{document}

% === 摘要页 ===
\begin{titlepage}
\begin{center}
\vspace*{2cm}

{\Large\textbf{MCM/ICM}}\\[0.5cm]
{\large Summary Sheet}\\[2cm]

{\Large Team Control Number: \controlnumber}\\[1cm]
{\large Problem Chosen: \problemtype}\\[2cm]

\begin{summary}
% === 在此处填写摘要 ===
[Your abstract goes here. This should be 300-500 words summarizing your approach 
and key findings. Include: background context, problem statement, methodology overview, 
key innovations, main results, and conclusions.]

\vspace{1em}
\textbf{Keywords:} keyword1; keyword2; keyword3; keyword4; keyword5
\end{summary}

\end{center}
\end{titlepage}

% === 目录(可选) ===
% \tableofcontents
% \newpage

% === 正文开始 ===
\section{Introduction}

\subsection{Problem Background}
% 问题背景描述

\subsection{Problem Restatement}
% 问题重述

\subsection{Our Approach}
% 方法概述

\section{Assumptions and Justifications}

\begin{itemize}
    \item \textbf{Assumption 1:} [Description]
    
    \textit{Justification:} [Why this assumption is reasonable]
    
    \item \textbf{Assumption 2:} [Description]
    
    \textit{Justification:} [Why this assumption is reasonable]
\end{itemize}

\section{Notations and Definitions}

\begin{table}[H]
\centering
\caption{Notation Summary}
\begin{tabular}{@{}cll@{}}
\toprule
\textbf{Symbol} & \textbf{Description} & \textbf{Unit} \\
\midrule
$x$ & Decision variable & - \\
$\alpha$ & Parameter & - \\
$N$ & Total count & units \\
\bottomrule
\end{tabular}
\end{table}

\section{Model Development}

\subsection{Model Overview}
% 模型概述

\subsection{Mathematical Formulation}
% 数学公式

The objective function is:
\begin{equation}
    \min_{x} f(x) = \sum_{i=1}^{n} c_i x_i
\end{equation}

Subject to:
\begin{align}
    \sum_{j=1}^{m} a_{ij} x_j &\leq b_i, \quad \forall i \\
    x_j &\geq 0, \quad \forall j
\end{align}

\subsection{Solution Approach}
% 求解方法

\section{Results and Analysis}

\subsection{Main Results}
% 主要结果

\begin{figure}[H]
\centering
% \includegraphics[width=0.8\textwidth]{figure1.pdf}
\caption{Main Results Visualization}
\label{fig:results}
\end{figure}

\subsection{Analysis}
% 结果分析

\section{Sensitivity Analysis}

\subsection{Parameter Sensitivity}
% 参数敏感性

\begin{table}[H]
\centering
\caption{Sensitivity Analysis Results}
\begin{tabular}{@{}lcc@{}}
\toprule
\textbf{Parameter} & \textbf{S1 (First-order)} & \textbf{ST (Total)} \\
\midrule
$\alpha$ & 0.45 & 0.52 \\
$\beta$ & 0.30 & 0.35 \\
$\gamma$ & 0.15 & 0.22 \\
\bottomrule
\end{tabular}
\end{table}

\subsection{Robustness}
% 鲁棒性分析

\section{Model Evaluation}

\subsection{Strengths}
\begin{itemize}
    \item \textbf{Strength 1:} [Description with evidence]
    \item \textbf{Strength 2:} [Description with evidence]
\end{itemize}

\subsection{Weaknesses}
\begin{itemize}
    \item \textbf{Weakness 1:} [Description] $\rightarrow$ [Mitigation]
    \item \textbf{Weakness 2:} [Description] $\rightarrow$ [Mitigation]
\end{itemize}

\subsection{Future Work}
% 未来工作

\section{Conclusions}
% 结论

% === 参考文献 ===
\bibliographystyle{plain}
% \bibliography{references}

\begin{thebibliography}{99}
\bibitem{ref1} Author, A. (2024). Title of Paper. \textit{Journal Name}, 10(1), 1-15.
\bibitem{ref2} Author, B., \& Author, C. (2023). Another Paper. \textit{Conference Name}.
\end{thebibliography}

% === 附录(如需要) ===
\appendix

\section{Supplementary Materials}
% 补充材料

\section{Code Listings}
% 代码清单

\end{document}
